\section*{List of Abbreviations}
\addcontentsline{toc}{section}{List of Abbreviations}
\begin{acronym}
%	\acro{asic}[ASIC]{Application Specific Integrated Circuit}
		
	\acro{gpdk}[GPDK]{Generic Process Design Kit}
		
	\acro{hdl}[HDL]{Hardware Description Language}
		
	\acro{rtl}[RTL]{Register-Transfer Level}
	
	\acro{tcl}[TCL]{Tool Command Language}
\end{acronym}


