% to compile: Shift+Alt+F1
\documentclass[a4paper,11pt]{article}%,twocolumn
\input{settings/packages}
\input{settings/page}
\input{settings/macros}


\begin{document}
\begin{titlepage}
\center % Center everything on the page

%-------------------------------------------------------------------------------------
%	HEADING SECTIONS
%------------------------------------------------------------------------------------
\textbf{\large Department of Electronic and Telecommunication Engineering}\\[0.5cm]
\textbf{\Large University of Moratuwa, Sri Lanka}\\[1cm]
\textbf{\large EN4603 - Digital IC Design}\\[2cm]
\includegraphics[width=0.3\textwidth]{figures/uomlogo}\\[2cm]

	
%-------------------------------------------------------------------------------------
%	TITLE SECTION
%------------------------------------------------------------------------------------
\textbf{\Huge Laboratory Experiment 1 \\RTL Synthesis }\\[0.2cm]
\textbf{\Large Laboratory Report}\\[3cm]


%----------------------------------------------------------------------------------------
%	MEMBERS SECTION
%----------------------------------------------------------------------------------------


\vfill

\textbf{\large Submitted by}\\[0.5cm]

\textbf{\large Submitted by}\\[2mm]
\begin{minipage}{0.3\textwidth}
	\begin{flushleft}
		{\large C.S.Pallikkonda}\\[2mm]
		{\large R.M.A.S.Rathnayake }\\[2mm]
		{\large B.P.Thalagala }\\[2mm]		
		
	\end{flushleft}
\end{minipage}
\hspace{2mm}
\begin{minipage}{0.2\textwidth}
	\begin{flushright}
		{\large 180441C }\\[2mm]
		{\large 180534N }\\[2mm]
		{\large 180631J }\\[2mm]

	\end{flushright}
\end{minipage}\\[1cm]

	
	
	
%----------------------------------------------------------------------------------------
%	DATE SECTION
%----------------------------------------------------------------------------------------

\textbf{\large Submitted on}\\[0.5cm]
\textbf{\Large \today} % Date, change the \today to a set date if you want to be precise

%----------------------------------------------------------------------------------------

\vfill % Fill the rest of the page with whitespace

\end{titlepage}

\pagebreak

\tableofcontents
\listoffigures
\listoftables
\vfill
\begin{center}
	\textbf{\textit{*PDF is clickable}}
\end{center}

\textit{\textbf{Note:}}\\
\textit{All the materials related to the report can also be found at \url{https://github.com/bimalka98/Digital-IC-Design}}
\pagebreak

\section{Introduction}

In this practical, we will be using Cadence Genus to synthesize an example RTL design: a
transceiver. As inputs to Genus, we will provide

\begin{enumerate}
	\item Source Verilog files
	\item Technology libraries provided by the fabrication plant (here, 35 nm educational GPDK
given by cadence) : .lib, .lef, .tch
	\item Timing constraints
\begin{enumerate}

and will obtain the synthesized netlist (Verilog files) and further timing constrains (.sdc) as
output. We will then analyze the area, timing and power of the synthesized design.

\section{Exercise}

\subsection{System Clocks \& Resets}

\subsubsection{System Clocks}





\vfill
\hrule
\vspace{0.5cm}
\bibliographystyle{plain}
\bibliography{refer}

%---------------------------------------------------------------------------
\end{document}
-
